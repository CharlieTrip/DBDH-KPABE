\chapter{Attributive Based Encription}

ABE (Attributive Based Encription) è una famiglia di schemi di crittografia basati tutti sull'utilizzo di una mappa bilineare $g:G_1 \times G_2 \rightarrow G_3$ con i $G_i$ gruppi ciclici di ordine un primo.

È possibile suddividere in due famiglie che si differenziano sul processo di autorizzazione delle chiavi di decifratura :
\begin{itemize}
	\item \textbf{{Chipertext Policy ABE}} :  Il messaggio cifrato contiene le regole d'autorizzazione. Questa viene concessa se la chiave private dell'utente soddisfa la policy d'autorizzazione del messaggio.
	\item \textbf{Key Policy ABE} : La chiave dell'utente contiene le informazioni sulla policy d'autorizzazione. Il messaggio cifrato conterrà gli attributi necessari per permettere la decodifica.
\end{itemize}

In entrambi i casi, possiamo caratterizzare\cite{maya3} queste famiglia in base ai gruppi su cui agisce la mappa bilineare :
\begin{itemize}
	\item $G_1 = G_2$ e di ordine $q$ primo
	\item $G_1 \neq G_2$ entrambi di ordine $q$ primo ed esiste un isomorfismo tra loro
	\item $G_1 \neq G_2$ entrambi di ordine $q$ primo e non esiste un isomorfismo tra loro
\end{itemize}	

Utilizzeremo il primo tipo dove $G_1 = G_2$ ed entrambi di ordine $q$ primo in uno schema \textbf{Key Policy ABE}


\section{Struttura generale}

Consideriamo un insieme di parties $\{P_1, \cdots , P_n \}$ su cui costruiamo una gerarchia $\mathbb{A}$ che raccoglie le varie tipologie d'accesso che vogliamo garantire e verranno descritte come sottoinsiemi di parties.\\
In questo modo, un insieme $A$ è autorizzato dal sistema unicamente se è contenuto nella gerarchia $\mathbb{A}$.

Nel nostro caso, il ruolo dei parties viene preso dagli attributi.


\begin{defi}
Un \textbf{struttura d'accesso} è una collezione di sottoinsiemi non vuoti di parties $\mathbb{A} \subseteq 2^{\{ P_1,\cdots,P_n \}} \setminus \{ \emptyset \}$.\\
Gli insiemi di $\mathbb{A}$ sono detti \textbf{insiemi autorizzati} e gli insiemi non in $\mathbb{A}$ sono detti \textbf{insiemi non autorizzati}.
\end{defi}

\begin{defi}
Una collezione $\mathbb{A} \subseteq 2^{\{P_1,\cdots,P_n\}}$ è detta \textbf{monotona} se per ogni $B,C$ sottoinsiemi di $\{ P_1,\cdots,P_n \}$ con $B \subseteq C$ e $B \in \mathbb{A}$ allora $C \in \mathbb{A}$.
\end{defi}

Nella nostra trattazione useremo sempre strutture d'accesso monotone.\\[0.8cm]




La struttura di uno schema KP-ABE prevede 4 algoritmi:
\begin{description}
\item[Impostazione :] Algoritmo con in ingresso unicamente i parametri di sicurezza impliciti del sistema. Ritorna le chiavi publiche \pk e la chiave principale \mk.
\item[Criptare :] Algoritmo che riceve un messaggio $m$, un insieme di attributi $\gamma$ e i parametri pubblici \pk. Ritorna il messaggio criptato $E$.
\item[Generazione chiavi :] Algoritmo che partendo da una struttura d'accesso $\mathbb{A}$, la chiave principale \mk e le chiavi pubbliche , fornisce una chiave di decifrazione $D$.
\item[Decifrare :]L'algoritmo riceve in input il messaggio criptato $E$ che è stato criptato in un insieme d'attributi $\gamma$, la chiave di decifrazione $D$ per l'accesso alla struttura $\mathbb{A}$ e i parametri pubblici \pk.\\
Ritornerà il messaggio $M$ se $\gamma \in \mathbb{A}$
\end{description}

\vspace{0.8cm}

Per poter valutare la sicurezza del sistema, si definisce un modello \textbf{selective-set (game)} che procede in questo modo:
\begin{description}
\item[Inizializzazione :] L'avversario \evil{E} dichiara l'insieme degli attributi $\gamma$ su cui vuole esser sfidato.
\item[Impostazione :] Lo sfidato \evil{C} esegue il setup del sistema KP-ABE e fornisce ad \evil{E} i parametri pubblici all'avversario.
\item[Fase 1 :] \evil{E} ha il permesso di interrogare \evil{C} per tutte le chiavi private delle strutture $\mathbb{A}_j$ dove $\gamma$ non è un accesso autorizzato.
\item[Sfida :] L'avversario \evil{E} invia due messaggi di ugual lunghezza $M_0$ ed $M_1$.\\
\evil{C} lancia una moneta binaria $b$ e cripta il messaggio $M_b$ con $\gamma$. Il cifrato è mandato a \evil{E}.
\item[Fase 2 :] Come fase 1.
\item[Guess :] \evil{E} ritorna una risposta $b^\prime$ rispetto a $b$
\end{description}

Mediante questo gioco, il vantaggio dell'avversario \evil{A} è definito come \[\left\lvert \mathbb{P}(b = b^\prime) - \frac{1}{2}\right\rvert\]

\begin{defi}
Uno schema KP-ABE è \textbf{sicuro} in un modello di sicurezza selective-set se ogni avversario a tempo polinomiale ha al massimo un vantaggio trascurabile nel gioco selective-set.
\end{defi}

dove definiamo
\begin{defi}
Tratto da \cite[Def~1.2]{crittoalice}.\\
Un avversario \evil{A} è detto \textbf{avversario a tempo polinomiale} se usa nella sua logica un certo grado di randomicità (\evil{A} è quindi probabilistico) e se l'esecuzione di \evil{A} termina in un numero di passi $t$ polinomiale rispetto alla lunghezza dell'input. 
\end{defi}

\vspace{0.5cm}
Considerando la mappa bilineare non degenere $e:G_1 \times G_1 \rightarrow G_2$, il nostro obbiettivo è di dimostrare

\begin{assu}[Decisionale Bilineare Diffie - Hellman]
Siano $a,b,c,z \in \mathbb{Z}_p$ scelti casualmente e sia $g$ un generatore di $G_1$ ciclico dove $p$ è l'ordine di $G_1$\\
Nessun algoritmo $\mathcal{B}$ a tempo polinomiale probabilistico può distinguere tra le tuple
\[ (A = g^a , B = g^b , C = g^c , e(g,g)^{abc}) \qquad (A = g^a , B = g^b , C = g^c , e(g,g)^{z}) \]
con al più un vantaggio trascurabile.\\
Il vantaggio di $\mathcal{B}$ è
\[ \left\lvert \mathbb{P}\left( \mathcal{B}(A,B,C,e(g,g)^{abc}) = 0 \right) - \mathbb{P}\left( \mathcal{B}(A,B,C,e(g,g)^{z}) = 0 \right) \right\rvert \]
dove la probabilità è \emph{considerata} sulla scelta casuale di $a,b,c,z \in \mathbb{Z}_p$ e la scelta casuale di $\mathcal{B}$.
\end{assu}

\subsection{Costruzione del sistema} 

% Le chiavi private sono identificate da una struttura d'accesso ad albero dove ogni nodo interno è una funzione di soglia e le foglie sono gli attributi. \\
Un utente potrà decifrare un messaggio cifrato con una data chiave privata unicamente se il messaggio soddisfa la policy d'accesso della chiave.\\[0.3cm]

\begin{defi}\-\\
Sia \evil{T} un \textbf{albero} rappresentante la nostra struttura d'accesso $\mathbb{A}$.\\[0.1cm]
Ogni non-foglia dell'albero rappresenta una funzione di soglia, descritta dai suoi figli e da un valore di soglia.\\
Se $\num_x$ è il numero di figli di un dato nodo $x$ e $k_x$ e il suo valore di soglia, allora $0 < k_x \leq \num_x$.\\
Quando $k_x = 1$, la funzione agisce come un OR mentre per $k_x = \num_x$ essa diventa una AND.\\
Ogni foglia è descritta dal suo attributo e una soglia di $k_x = 1$.\\[0.2cm]
Nel nostro albero ci è possibile definire un ordine rispetto ai figli di un nodo $x$ che li ordina da $1$ a $\num_x$. Questi indici sono univocamente assegnati ai nodi per ogni chiave d'accesso arbitraria.
\end{defi}
\vspace{0.3cm}

Per semplificare il lavoro con gli alberi, definiamo delle funzioni:
\begin{itemize}
\item $\parent(x)$ : ci ritorna il genitore del nodo $x$.
\item $\att(x)$ : se $x$ è una foglia, ci fornisce il suo attributo.
\item $\inde(x)$ : fornisce l'indice dell'ordine del nodo $x$ rispetto al genitore $\parent(x)$
\item $\mathcal{T}_x$ : è il sottoalbero di \evil{T} con radice $x$ 
\end{itemize}

\vspace{0.3cm}
A questo punto, creiamo un algoritmo che ci permetta di verificare se un dato insieme di attributi può accedere o meno nell'albero.
\vspace{0.5cm}

\textbf{ \large Soddisfare un albero d'accesso}\\[0.1cm]
Denotiamo con 
\begin{center}
$\mathcal{T}_x(\gamma) = 1$ se e solo se l'insieme d'attributi $\gamma$ soddisfa l'albero d'accesso $\mathcal{T}_x$
\end{center}
Calcoliamo $\mathcal{T}_x(\gamma)$ in questo modo:
\begin{description}
\item[$x$ è una foglia :] \[\mathcal{T}_x(\gamma) = 1 \Leftrightarrow att(x) \in \gamma\]
cioè è garantito l'accesso se gli attributi della foglia sono presenti in $\gamma$
\item[$x$ non è una foglia :] \[\mathcal{T}_x(\gamma) = 1  \Leftrightarrow \# \{x^\prime \text{ figli di } x \quad|\quad \mathcal{T}_{x^\prime}(\gamma) = 1  \} \geq k_x\]
cioè viene garantito l'accesso unicamente se il numero di figli che autorizzano $\gamma$ sono almeno in numero la soglia del nodo $x$ 
\end{description}

\vspace{0.8cm}

Ora prendiamo un $G_1$ gruppo bilineare di ordine $p$ e sia $g$ un generatore di $G_1$.\\
%Un parametro di sicurezza $k$ determinerà la grandezza dei gruppi.\\ % Non è utile nella tesi e può esser richiamato
\begin{defi}\label{pairinge}
$e : G_1 \times G_1 \rightarrow G_2$ mappa bilineare che soddisfa :
\begin{enumerate}
\item $e$ è bilineare rispetto al prodotto
\[ \forall_{u,v \in G_1}, a,b \in \mathbb{Z}_p : e(u^a,v^b) = e(u,v)^{ab} \]
\item $e$ è non degenere \[e(g,g) \neq 1\]
\item $e$ è velocemente computabile
\end{enumerate}
\end{defi}

\vspace{0.8cm}

\textbf{\Large Creazione sistema}

\begin{description}
\item[Impostazione :]Definiamo l'universo di attributi $\mathcal{U} = \{1,\cdots,n\}$. Ora, per ogni attributo $i \in \mathcal{U}$, scegliamo uniformemente a caso un elemento $t_i$ di $\mathbb{Z}_p$. Allo stesso modo prendiamo un $y$.\\
I parametri pubblici \pk pubblicabili sono
\[ T_1 = g^{t_1} , T_2 = g^{t_2} , \cdots , T_n = g^{t_n} , Y = e(g,g)^y \]\label{pubchiavi}
mentre la chiave principale \mk è
\[ t_1 , t_2 , \cdots , t_n , y \]
\vspace{0.1cm}
\item[Criptare :] Partendo da $(M , \gamma, \text{ \pk })$, cripto $M \in G_2$ sotto un insieme di attributi $\gamma$ scegliendo un $s$ casualmente in $\mathbb{Z}_p$ e pubblico il testo cifrato come
\[ E = (\gamma , E^\prime = MY^s , \{E_i = T_i^s \}_{i\in\gamma}) \]
\vspace{0.1cm}
\item[Generazione chiavi :] Partendo da $( \mathcal{T} , MK )$, l'algoritmo ci fornisce una chiave capace di decifrare il messaggio rispetto a $\gamma$ se e solo se $\mathcal{T}(\gamma) = 1$ cioè $\gamma$ è un insieme di attributi autorizzato.
\begin{enumerate}
\item Scelgo un polinomio $q_x$ per ogni nodo $x$, incluse le foglie, dell'albero \evil{T}.\\
I polinomi sono scelti dall'alto al basso, partendo quindi dalla radice.
\item Per ogni nodo $x$, sia $d_x$ il grado del polinomio $q_x$ tale che $d_x = k_x - 1$ con $k_x$ valore di soglia del nodo.
\item Per il nodo principale $r$, fissiamo $q_r(0) = y$ ed altri $d_r$ punti per completare la definizione del polinomio $q_r$.
\item Per gli altri nodi, fissiamo $q_x(0) = q_{\parent(x)}(\inde(x))$ e scegliamo altri $d_x$ altri punti per completare il polinomio.
\item Per ogni foglia $x$, ritorniamo all'utente la chiave di decriptazione
\[ D_x = g^{\frac{q_x(0)}{t_i}} \qquad \text{ dove } i \in \att(x) \]
\end{enumerate}
L'insieme di tutte le chiavi di sopra forma la chiave di decriptazione $D$
\vspace{0.1cm}
\item[Decifrare :] Partendo da $(E,D)$. Definiamo il nostro metodo di decifrazione mediante un algoritmo ricorsivo:\\
Sia $\decrypt( E , D , x )$ in un elemento di $G_2$ oppure non è permesso calcolarlo e lo indicheremo con $\perp$.\\
Sia $i = \att(x)$.
\begin{itemize}
\item Se $x$ è una foglia
\[ \decrypt(E,D,x) = \begin{cases}
\perp \qquad \text{ se } i \notin \gamma\\
e(D_x,E_i) = e(g^{\frac{q_x(0)}{t_i}},g^{s\cdot t_i}) = e(g,g)^{q_x(0) \cdot s}
\end{cases} \]
\item Se $x$ non è una foglia :\\
Per ogni nodo figlio $z$ di $x$, $F_z = \decrypt(E,D,z)$.\\
Considero i coefficienti di Lagrange :
\[ \Delta_{i,S}(x) = \prod_{j \in S , j \neq i} \frac{x-j}{i-j} \]

Sia $S_x$\label{insiemedecript} un insieme arbitrario di dimensione $k_x$ di figli $z$ tali che $F_z \neq \perp$.\\
Se non esiste alcun insieme di questo tipo, allora la funzione ritorna $\perp$. Altrimenti
\begin{align*}
F_x &= \prod_{z \in S_x} F_z^{\Delta_{i,S_x^\prime}(0)} \qquad \text{ dove }
\begin{matrix}
i = \inde(x)\\
S^\prime_x = \{ \inde(z) : z \in S_x \}
\end{matrix}\\
&= \prod_{z \in S_x} (e(g,g)^{s \cdot q_z(0)})^{\Delta_{i,S_x^\prime}(0)}\\
&= \prod_{z \in S_x} (e(g,g)^{s \cdot q_{\parent(z)}(\inde(z))})^{\Delta_{i,S_x^\prime}(0)} \qquad \text{(per costruzione)}\\
&= \prod_{z \in S_x} e(g,g)^{s \cdot q_x(i) \cdot \Delta_{i,S_x\prime}(0)}\\
&= e(g,g)^{s \cdot q_x(0)} 
\end{align*}
Dove l'ultimo passaggio è stato svolto per interpolazione polinomiale:\\
consideriamo il polinomio
\[ P(t) = \sum_{i \in S^\prime_x} q_x(i) \Delta_{i,S^\prime_x}(t)\]
Notiamo che la cardinalità dell'insieme $ S^\prime_x$ è maggiore rispetto al grado di $q_x(t)$.\\
La valutazione di $P(t)$ nei punti di $S^\prime_x$ ci fornisce 
\[P(j) = \sum_{i \in S^\prime_x} q_x(i) \Delta_{i,S^\prime_x}(j) = \sum_{i \in S^\prime_x} q_x(i) \delta_{i,j} = q_x(j)\]
cioé la valutazione di $P$ corrisponde con quella di $q_x$ in esattamente $\# S^\prime_x = \deg q_x + 1$ punti.\\
Per questo, i due polinomi coincidono ed abbiamo $P(0) = q_x(0)$.
% \href{http://math.stackexchange.com/questions/685472/find-n-degree-polynomial-from-n1-points}{(Spiegazione formale)}

\item Per decifrare usiamo $\decrypt (E,D,r)$ dove $r$ è la radice dell'albero
\end{itemize}
Quel che si nota è che se soddisfiamo l'albero, $\decrypt (E,D,r) = e(g,g)^{ys} = Y^s$ e poiché $E^\prime = MY^s$, l'algoritmo semplicemente divide per $Y^s$ che ottiene dal calcolo e recupera il messaggio $M$.
\end{description}





\subsection{Assunzione Diffie - Hellman}

\begin{thm}
Se un avversario può rompere lo schema in un modello ABE Selective-Set, allora un simulatore può esser costruito per giocare al gioco Decisional Bilinear Diffie Hellman con un vantaggio non trascurabile.
\begin{proof}
Per dimostrare il teorema, supponiamo esista un avversario \evil{A} a tempo polinomiale che può rompere lo schema con un vantaggio $\epsilon$.\\
Costruiamo quindi un simulatore \evil{B} capace di giocare al gioco D-BDH con un vantaggio di $\epsilon / 2$ in questo modo :\\[0,5cm]

Facciamo scegliere allo sfidante i gruppi $G_1, G_2$ con una mappa $e$ efficente e un generatore $g$.\\
Successivamente il simulatore lancia una moneta $\mu$ di cui risultato non viene rivelato allo sfidante.\\
Se $ \mu = 0$ allora viene scelto $(A,B,C,Z) = (g^a,g^b,g^c,e(g,g)^{abc})$ altrimenti $(A,B,C,Z) = (g^a,g^b,g^c, e(g,g)^z)$ per dei valori $a,b,c,z$ casuali.\\
Assumiamo l'insieme degli attributi \evil{U} definito.

\begin{description}
\item[Inizializzazione :] Il simulatore \evil{B} esegue \evil{A}. \evil{A} decide l'insieme degli attributi $\gamma$ su cui vuol esser sfidato.
\item[Setup :] Il simulatore \evil{B} fissa il parametro $Y= e(A,B) = e(g,g)^{ab}$.\\
Per ogni $i \in \mathcal{U}$, viene fissato $T_i$ come :
\begin{enumerate}
\item Se $i \in \gamma$ allora si sceglie casualmente $r_i \in \mathbb{Z}_p$ e fissiamo $T_i = g^{r_i}$
\item Altrimenti scegliamo casualmente $\beta_i$ e fissiamo $T_i = g^{\beta_i b} = B^{\beta_i}$
\end{enumerate}
Successivamente \evil{A} ritorna i parametri pubblici a \evil{B}
\item[Fase 1 :] \evil{A} esegue richieste di chiavi corrispondenti ad ogni albero d'accesso \evil{T} tale che $\gamma$ non soddisfa \evil{T}.\\
Supponiamo che \evil{A} faccia una richiesta per un albero \evil{T} dove $\mathcal{T}(\gamma) = 0$.\\
Per generare la chiave, \evil{B} deve assegnare un polinomio $Q_x$ di grado $d_x$ per ogni nodo $x$ di \evil{T}.\\
Definiamo quindi due casistiche :
\begin{itemize}
\item $\polysat (\mathcal{T}_x , \gamma, \lambda_x )$ : questa procedura viene eseguita se $\mathcal{T}_x(\gamma) = 1$ cioè viene soddisfatto l'albero. $\lambda_x$ è un valore in $\mathbb{Z}_p$.\\
Per prima cosa fissiamo un polinomio $q_x$ di grado $d_x$ per il nodo radice $x$ tale che $q_x(0) = \lambda_x$ e setta in maniera casuale gli altri valori per completare $q_x$.\\
Successivamente, per ogni figlio $x^\prime$ di $x$, viene richiamata la procedura $\polysat(\mathcal{T}_{x^\prime}, \gamma, q_x(\inde(x^\prime))$ così da ottenere $q_{x^\prime} (0) = q_x (\inde(x^\prime))$.
\item $\polyunsat(\mathcal{T}_x, \gamma , g^{\lambda_x})$ : questa procedura viene eseguita se $\mathcal{T}_x (\gamma) = 0$ cioé quando non viene soddisfatto l'albero. $g^{\lambda_x} \in \mathbb{G}_1$.\\
Viene fissato un polinomio $q_x$ di grado $d_x$ per il nodo $x$ tale che $q_x(0) = \lambda_x$. Siccome $\mathcal{T}_x(\gamma) = 0$, non più di $d_x$ figli soddisfano l'albero.\\
Sia quindi $h_x \leq d_x$ il numero di figli che soddisfano l'albero in $x$ e per questi viene scelto a caso un $\lambda_{x^\prime} \in \mathbb{Z}_p$ e fissiamo $q_{x^\prime}(0) = \lambda_{x^\prime}$. Successivamente vengono fissati casualmente i restanti $d_x - h_x$ punti di $q_x$ per completare il polinomio.\\
L'algoritmo procede nella definizione dei polinomi per ogni nodo figlio $x^\prime$ di $x$ come
\begin{itemize}
\item $\polysat(\mathcal{T}_{x^\prime} , \gamma , q_x(\inde(x^\prime)))$ se $x^\prime$ è un nodo soddisfacente $\gamma$.\\
Da osservare che in questo caso siamo a conoscenza di $q_x(\inde(x^\prime))$.
\item $\polyunsat(\mathcal{T}_{x^\prime} , \gamma , g^{q_x(\inde(x^\prime))})$ se $x^\prime$ non è un nodo soddisfacente $\gamma$.\\
Da osservare che in questo caso siamo a conoscenza unicamente di $g^{q_x(\inde(x^\prime))}$ per interpolazione a partire da $g^{q_x(0)}$.
\end{itemize}
Osserviamo che anche in questo caso $q_{x^\prime}(0) = q_x(\inde (x^\prime))$ per ogni nodo figlio $x^\prime$ di $x$.
\end{itemize}
Per fornire gli accessi alla struttura \evil{T}, il simulatore esegue $\polyunsat(\mathcal{T},\gamma,A)$ per definire un polinomio $q_x$ per ogni nodo $x$ di \evil{T}.\\
Per ogni foglia $x$ di \evil{T} conosciamo $q_x$ completamente se $x$ soddisfa $\gamma$, altrimenti conosciamo unicamente $g^{q_x(0)}$.\\
Inoltre $q_r(0) = a$.\\
Il simulatore ora definisce $Q_x(\cdot) = b q_x( \cdot)$ per ogni nodo $x \in \mathcal{T}$.\\
Osserviamo che così otteniamo $y = Q_r(0) = ab$. La chiave corrispondente per ogni foglia è data utilizzando il suo polinomio come segue. Sia $i = \att(x)$.
\[ D_x = \begin{cases}
g^{\frac{Q_x(0)}{t_i}} = g^{\frac{b q_x(0)}{r_i}} = B^{\frac{q_x(0)}{r_i}}  \qquad \text{ se } i \in \gamma\\
g^{\frac{Q_x(0)}{t_i}} = g^{\frac{b q_x(0)}{b \beta_i}} = g^{\frac{q_x(0)}{\beta_i}} \qquad \text{ altrimenti}
\end{cases} \]
In questo modo il simulatore \evil{B} riesce a costruire le chiavi private per gli accessi alla struttura \evil{T}.
\item[Sfida :] L'avversario \evil{A} invia due messaggi $m_0$ e $m_1$ al simulatore \evil{B}.\\
\evil{B} lancia una moneta binaria equiprobabile $\nu$ e ritorna il cifrato di $m_\nu$ come
\[ E = (\gamma , E^\prime = m_\nu Z , \{ E_i = C^{r_i} \}_{i \in \gamma} ) \]
Se $\mu = 0$ allora $Z = e(g,g)^{abc}$.\\
Da questo abbiamo che $s = c$ cioé $Y^s = (e(g,g)^{ab})^c = e(g,g)^{abc}$ e $E_i = (g^{r_i})^c = C^{r_i}$. Otteniamo quindi una valida cifratura del messaggio $m_\nu$.\\
Se $\mu = 1$ allora $E^\prime = m_\nu e(g,g)^z$ è un elemento casuale di $\mathbb{G}_2$ e non è così possibile avere informazioni su $m_\nu$ data l'arbitrarietà di $z$.
\item[Fase 2 :] Viene ripetuto quel che è stato fatto nella Fase 1.
\item[Guess :] \evil{A} manda un ipotesi $\nu^\prime$ di $\nu$. Se $\nu^\prime = \nu$ allora \evil{B} ritornerà $\mu^\prime = 0$ per indicare che è stata data una BDH-tupla valida, $\mu^\prime = 1$ se la tupla è casuale.
\end{description}

Come si può vedere, la costruzione del simulatore è uguale a quella dello schema originale.\\[0.5cm]

Calcoliamo ora il vantaggio dello sfidante nel selective-set game che risulta essere
\[\mathbb{P}[ \mu = \mu^\prime] - \frac{1}{2} \]
Nel caso $\mu = 1$, l'avversario non riceve informazioni su $\nu$.
Quindi abbiamo che $\mathbb{P}[\nu \neq \nu^\prime | \mu = 1] = \frac{1}{2}$.
Siccome \evil{A} cerca di indovinare $\mu^\prime = 1$ quando $\nu^\prime \neq \nu$, abbiamo che $\mathbb{P}[\mu = \mu^\prime| \mu = 1] = \frac{1}{2}$\\[0.2cm]
Se $\mu = 0$, \evil{A} vede il cifrato di $m_\nu$. Il vantaggio è $\epsilon$ dà ipotesi.\\
Quindi $\mathbb{P}[\nu \neq \nu^\prime | \mu = 0] = \frac{1}{2} + \epsilon$.\\
Siccome viene ipotizzato $ \mu = \mu^\prime$ quando $\nu \neq \nu^\prime$, abbiamo $\mathbb{P}[\mu = \mu^\prime | \mu = 0] = \frac{1}{2} + \epsilon$.\\[0.2cm]
A questo punto, il vantaggio complessivo di un simulatore che gioca al DBDH è 
\[ \frac{1}{2} \mathbb{P}[\mu = \mu^\prime | \mu = 1] + \frac{1}{2}\mathbb{P}[\mu = \mu^\prime | \mu = 0] - \frac{1}{2}  = \frac{1}{2}\left( \frac{1}{2} + \epsilon \right) + \frac{1}{2}\frac{1}{2} - \frac{1}{2} = \frac{1}{2} \epsilon\]
\end{proof}
\end{thm}



