\subsection{Costruzione del sistema} 

% Le chiavi private sono identificate da una struttura d'accesso ad albero dove ogni nodo interno è una funzione di soglia e le foglie sono gli attributi. \\
Un utente potrà decifrare un messaggio cifrato con una data chiave privata unicamente se il messaggio soddisfa la policy d'accesso della chiave.\\[0,5cm]

\begin{defi}
Sia \evil{T} un \textbf{albero} rappresentante la nostra struttura d'accesso $\mathbb{A}$.\\
Ogni non-foglia dell'albero rappresenta una funzione di soglia, descritta dai suoi figli e da un valore di soglia.\\
Se $\num_x$ è il numero di figli di un dato nodo $x$ e $k_x$ e il suo valore di soglia, allora $0 < k_x \leq \num_x$.\\
Quando $k_x = 1$, la funzione agisce come un OR mentre per $k_x = \num_x$ essa diventa una AND.\\
Ogni foglia è descritta dal suo attributo e una soglia di $k_x = 1$.\\
Nel nostro albero ci è possibile definire un ordine rispetto ai figli di un nodo $x$ che li ordina da $1$ a $\num_x$. Questi indici sono univocamente assegnati ai nodi per ogni chiave d'accesso arbitraria.
\end{defi}


Per semplificare il lavoro con gli alberi, definiamo delle funzioni:
\begin{itemize}
\item $\parent(x)$ : ci ritorna il genitore del nodo $x$.
\item $\att(x)$ : se $x$ è una foglia, ci fornisce il suo attributo.
\item $\inde(x)$ : fornisce l'indice dell'ordine del nodo $x$ rispetto al genitore $\parent(x)$
\item $\mathcal{T}_x$ : è il sottoalbero di \evil{T} con radice $x$ 
\end{itemize}

\vspace{0.8cm}
A questo punto, creiamo un algoritmo che ci permetta di verificare se un dato insieme di attributi può accedere o meno nell'albero.

\vspace{0.2cm}
\textbf{Soddisfare un albero d'accesso}\\
Denotiamo con 
\begin{center}
$\mathcal{T}_x(\gamma) = 1$ se e solo se l'insieme d'attributi $\gamma$ soddisfa l'albero d'accesso $\mathcal{T}_x$
\end{center}
Calcoliamo $\mathcal{T}_x(\gamma)$ in questo modo:
\begin{description}
\item[$x$ è una foglia :] \[\mathcal{T}_x(\gamma) = 1 \Leftrightarrow att(x) \in \gamma\]
cioè è garantito l'accesso se gli attributi della foglia sono presenti in $\gamma$
\item[$x$ non è una foglia :] \[\mathcal{T}_x(\gamma) = 1  \Leftrightarrow \# \{x^\prime \text{ figli di } x \quad|\quad \mathcal{T}_{x^\prime}(\gamma) = 1  \} \geq k_x\]
cioè viene garantito l'accesso unicamente se il numero di figli che autorizzano $\gamma$ sono almeno in numero la soglia del nodo $x$ 
\end{description}

\vspace{0.8cm}

Ora prendiamo un $G_1$ gruppo bilineare di ordine $p$ e sia $g$ un generatore di $G_1$.\\
%Un parametro di sicurezza $k$ determinerà la grandezza dei gruppi.\\ % Non è utile nella tesi e può esser richiamato
In aggiunta consideriamo $e : G_1 \times G_1 \rightarrow G_2$ mappa bilineare che soddisfa :
\begin{enumerate}
\item $e$ è bilineare rispetto al prodotto
\[ \forall_{u,v \in G_1}, a,b \in \mathbb{Z}_p : e(u^a,v^b) = e(u,v)^{ab} \]
\item $e$ è non degenere \[e(g,g) \neq 1\]
\item $e$ è velocemente computabile
\end{enumerate}


\vspace{0.8cm}

\textbf{Creazione sistema}

\begin{description}
\item[Impostazione :]Definiamo l'universo di attributi $\mathcal{U} = \{1,\cdots,n\}$. Ora, per ogni attributo $i \in \mathcal{U}$, scegliamo uniformemente a caso un elemento $t_i$ di $\mathbb{Z}_p$. Allo stesso modo prendiamo un $y$.\\
I parametri pubblici \pk pubblicabili sono
\[ T_1 = g^{t_1} , T_2 = g^{t_2} , \cdots , T_n = g^{t_n} , Y = e(g,g)^y \]\label{pubchiavi}
mentre la chiave principale \mk è
\[ t_1 , t_2 , \cdots , t_n , y \]
\item[Criptare :] Partendo da $(M , \gamma, \text{ \pk })$, cripto $M \in G_2$ sotto un insieme di attributi $\gamma$ scegliendo un $s$ casualmente in $\mathbb{Z}_p$ e pubblico il testo cifrato come
\[ E = (\gamma , E^\prime = MY^s , \{E_i = T_i^s \}_{i\in\gamma}) \]
\item[Generazione chiavi :] Partendo da $( \mathcal{T} , MK )$, l'algoritmo ci fornisce una chiave capace di decifrare il messaggio rispetto a $\gamma$ se e solo se $\mathcal{T}(\gamma) = 1$ cioè $\gamma$ è un insieme di attributi autorizzato.
\begin{enumerate}
\item Scelgo un polinomio $q_x$ per ogni nodo $x$, incluse le foglie, dell'albero \evil{T}.\\
I polinomi sono scelti dall'alto al basso, partendo quindi dalla radice.
\item Per ogni nodo $x$, sia $d_x$ il grado del polinomio $q_x$ tale che $d_x = k_x - 1$ con $k_x$ valore di soglia del nodo.
\item Per il nodo principale $r$, fissiamo $q_r(0) = y$ ed altri $d_r$ punti per completare la definizione del polinomio $q_r$.
\item Per gli altri nodi, fissiamo $q_x(0) = q_{\parent(x)}(\inde(x))$ e scegliamo altri $d_x$ altri punti per completare il polinomio.
\item Per ogni foglia $x$, ritorniamo all'utente la chiave di decriptazione
\[ D_x = g^{\frac{q_x(0)}{t_i}} \qquad \text{ dove } i \in \att(x) \]
\end{enumerate}
L'insieme di tutte le chiavi di sopra forma la chiave di decriptazione $D$
\item[Decifrare :] Partendo da $(E,D)$.\\
Definiamo il nostro metodo di decifrazione mediante un algoritmo ricorsivo:\\
Sia $\decrypt( E , D , x )$ in un elemento di $G_2$ oppure non è permesso calcolarlo e lo indicheremo con $\perp$. Sia $i = \att(x)$.
\begin{itemize}
\item Se $x$ è una foglia
\[ \decrypt(E,D,x) = \begin{cases}
\perp \qquad \text{ se } i \notin \gamma\\
e(D_x,E_i) = e(g^{\frac{q_x(0)}{t_i}},g^{s\cdot t_i}) = e(g,g)^{q_x(0) \cdot s}
\end{cases} \]
\item Se $x$ non è una foglia :\\
Per ogni nodo figlio $z$ di $x$, $F_z = \decrypt(E,D,z)$. Sia $S_x$\label{insiemedecript} un insieme arbitrario di dimensione $k_x$ di figli $z$ tali che $F_z \neq \perp$.\\
Se non esiste alcun insieme di questo tipo, allora la funzione ritorna $\perp$.\\
Altrimenti
\begin{align*}
F_x &= \prod_{z \in S_x} F_z^{\Delta_{i,S_x^\prime}(0)} \qquad \text{ dove }
\begin{matrix}
i = \inde(x)\\
S^\prime_x = \{ \inde(z) : z \in S_x \}
\end{matrix}\\
&= \prod_{z \in S_x} (e(g,g)^{s \cdot q_z(0)})^{\Delta_{i,S_x^\prime}(0)}\\
&= \prod_{z \in S_x} (e(g,g)^{s \cdot q_{\parent(z)}(\inde(z))})^{\Delta_{i,S_x^\prime}(0)} \qquad \text{(per costruzione)}\\
&= \prod_{z \in S_x} e(g,g)^{s \cdot q_x(i) \cdot \Delta_{i,S_x\prime}(0)}\\
&= e(g,g)^{s \cdot q_x(0)} 
\end{align*}
Dove l'ultimo passaggio è stato svolto per interpolazione polinomiale:\\
consideriamo il polinomio
\[ P(t) = \sum_{i \in S^\prime_x} q_x(i) \Delta_{i,S^\prime_x}(t)\]
Notiamo che la cardinalità dell'insieme $ S^\prime_x$ è maggiore rispetto al grado di $q_x(t)$.\\
La valutazione di $P(t)$ nei punti di $S^\prime_x$ ci fornisce 
\[P(j) = \sum_{i \in S^\prime_x} q_x(i) \Delta_{i,S^\prime_x}(j) = \sum_{i \in S^\prime_x} q_x(i) \delta_{i,j} = q_x(j)\]
cioé la valutazione di $P$ corrisponde con quella di $q_x$ in esattamente $\# S^\prime_x = \deg q_x + 1$ punti.\\
Per questo, i due polinomi coincidono ed abbiamo $P(0) = q_x(0)$.	\href{http://math.stackexchange.com/questions/685472/find-n-degree-polynomial-from-n1-points}{(Spiegazione formale)}

\item Per decifrare usiamo $\decrypt (E,D,r)$ dove $r$ è la radice dell'albero
\end{itemize}
Quel che si nota è che se soddisfiamo l'albero, $\decrypt (E,D,r) = e(g,g)^{ys} = Y^s$ e poiché $E^\prime = MY^s$, l'algoritmo semplicemente divide per $Y^s$ che ottiene dal calcolo e recupera il messaggio $M$.
\end{description}


