\chapter{Esempio Concreto}

Vogliamo costruire uno schema KP-ABE \textbf{funzionante e utilizzando strumenti molto semplici (non saprei come spiegare questa cosa)}.\\
Suddivideremo la costruzione nei vari passaggi più delicati per analizzarli in maniera più precisa.

\section{Mappa bilineare e gruppi}

Scegliamo il gruppo ciclico $\mathbb{Z}_{p}$ con l'operazione $+$ e definiamo la mappa bilineare come:
\[ e(g^a,g^b) = 2^{ab} = (ab) \times 2 : = 2 + 2 + \cdots + 2 \text{ $ab$ volte}\]

Dove per $g^a$ intendiamo $a$ volte $g$.\\
In questo modo abbiamo che
\[ D^{\frac{1}{n}} = \frac{1}{n} \times D = (p - n) \times D = p - n \times D\]
\vspace{1cm}


\begin{itemize}
	\item Si può parlare del pairing di Weil e Tate anche se sforano entrambi sulle curve elittiche per costruire il pairing
\end{itemize}

\section{Generazione e distribuzione delle chiavi}



Per la generazione delle chiavi private, è stato utilizzato il seguente algoritmo:
\vspace{0.3cm}
\begin{algorithmic}
\Function{generaPolinomi}{nodo dell'albero $x$, valore $v$}
   \If{$x$ non ha figli}
   		\State Salva il valore $v$ nel nodo.
   \Else
   		\State Sia $d = $ soglia di $x$ - 1
   		\If{ $d > 0$}
   			\State Definisco $q_x(t) = t^d + v$ \Comment{In questo modo $q_x(0) = v$}
   		
   		\Else
   			\State Definisco $q_x(t) = v$
   		
   		\EndIf

   		\ForAll{ $y$ figlio di $x$}
   			\State {\scshape generaPolinomi}($y$, $q_x(\inde(y))$) \Comment{Così otteniamo $q_y(0) = q_x(\inde(y))$}
   		\EndFor
   \EndIf
\EndFunction
\end{algorithmic}
\vspace{0.3cm}
In questo modo, richiamando {\scshape generaPolinomi}($r$ radice dell'albero, $s$) otteniamo che $q_x(0) = q_{\parent(x)}(\inde(x))$.

\section{Visione d'insieme}

