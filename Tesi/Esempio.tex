\chapter{Un esempio concreto}

\section{Mappa bilineare e gruppi}

Scegliamo il gruppo ciclico $\mathbb{Z}_{p}$ con l'operazione $+$ e definiamo la mappa bilineare come:
\[ e(g^a,g^b) = h^{ab} = (ab) \times h : = h + h + \cdots + h \text{ $ab$ volte}\]

Dove per $g^a$ intendiamo $a$ volte $g$.\\


Un problema è la necessità di conoscere il valore $a$ tale che $g^a$ è il nostro valore scelto. Questo risulta calcolare il logaritmo discreto che deve risultare difficile da calcolare per esser sicuro.\\
Questo pairing è quindi banale poiché la complessità del logaritmo discreto $h^a$ è equivalente ad eseguire l'algoritmo di Euclide esteso per risolvere $a \times h + j \times p = k \times h^a$ trovando $j,k$.\\


Per utilizzi più sicuri e concreti vengono utilizzate curve ellittiche su campi finiti. Su queste viene creato un pairing sfruttando caratteristiche proprie delle curve ellittiche.\\
È possibile trovare informazioni a riguardo sul pairing di Weil e Tate con algoritmi per rendere il calcolo del pairing efficente in \cite{benoit} e \cite{maas}.


\section{Generazione e distribuzione delle chiavi}

Per la generazione delle chiavi private, è stato utilizzato il seguente algoritmo:
\vspace{0.3cm}
\begin{algorithmic}
\Function{generaPolinomi}{nodo dell'albero $x$, valore $v$}
   \If{$x$ non ha figli}
   		\State Salva il valore $v$ nel nodo.
   \Else
   		\State Sia $d = $ soglia di $x$ - 1
   		\If{ $d > 0$}
   			\State Definisco $q_x(t) = t^d + v$ \Comment{{\small In questo modo $q_x(0) = v$}}

   		\Else
   			\State Definisco $q_x(t) = v$
   		
   		\EndIf

   		\ForAll{ $y$ figlio di $x$}
   			\State {\scshape generaPolinomi}($y$, $q_x(\inde(y))$) 
            \State {\itshape\small Così otteniamo $q_y(0) = q_x(\inde(y))$}
   		\EndFor
   \EndIf
\EndFunction
\end{algorithmic}
\vspace{0.3cm}
In questo modo, richiamando {\scshape generaPolinomi}($r$ radice dell'albero, $s$) otteniamo che $q_x(0) = q_{\parent(x)}(\inde(x))$.\label{polinomi}\\[1cm]

Osserviamo che l'effettiva chiave privata che viene data all'utente è della forma $D = g^{\frac{q_x(0)}{t_i}}$ dove ci focaliziamo su $t_i$.\\
In questa situazione, dobbiamo trovare la radice $t_i$-esima di $g$. Per rendere efficente l'utilizzo del sistema, è necessario tener conto di questo \emph{``problema''}.\\
Nell'esempio descritto in questa tesi, il calcolo della radice viene eseguito mediante l'algoritmo esteso di Euclide:

\begin{center}
\begin{tikzcd}
\text{Trovare } x = g^{\frac{1}{b}} \arrow{r} & \begin{tabular}{@{}c@{}} Risolvere \\$x^b = g $ \\  cioè  \end{tabular} \quad bx = g 
\arrow{l} \arrow{dr}& \- \\
\left. \begin{tabular}{@{}c@{}}Euclide \\ $u_1,u_2 : u_1 b + u_2 p = 1$  \\ modulo $p$ \end{tabular}\right\rvert \quad u_1 b = 1\arrow{r} & u_1 b x = u_1 g  \arrow{r} & x = u_1 g
\end{tikzcd}
\end{center}

In questo modo troviamo la radice in una maniera efficiente.



\section{Visione d'insieme}


Per come è descritto il sistema, l'utente \textbf{non} è a conoscenza del gruppo $G_1$ su cui è definita la mappa bilineare ma conosce unicamente $G_2$ su cui esegue le operazioni per decrittare i messaggi.\\
Per questo motivo un possibile attacco risulta molto articolato:\label{security}
\begin{enumerate}
   \item\textbf{ Logaritmo discreto su $G_2$}\\
   Si ricava il generatore $h$.
   \item \textbf{Mappa bilineare $e$}\\
   Conoscendo $h$, è sufficente trovare nell'insieme degli $e(a,a)$ quello che risulta uguale ad $h$. Questo è dovuto alle proprietà della mappa bilineare.\\
   In questo modo riusciamo ad ottenere $g$ generatore di $G_1$ e tale per cui $e(g,g) = h$.
   \item\textbf{ Logaritmo discreto su $G_1$}\\
   Ottenuto il generatore $g$, si può risolvere il logaritmo $g^x = T_i$ dove le $T_i$ sono le chiavi pubbliche dei vari attributi.\\
   A questo punto è possibile ricavare il valore $s$ (scelto casualmente nell'algoritmo di codifica) per poter decifrare il messaggio.
\end{enumerate}

È da notare come l'ultimo punto può risultare molto complesso se non si conosce l'operazione di $G_1$.\\
Nel nostro caso, un possibile attaccante può facilmente calcolare il logaritmo di $g^a$ poiché è unicamente necessario l'algoritmo di Euclide.\\[0.5cm]
