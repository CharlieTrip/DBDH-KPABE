\chapter{Conclusioni}



\section{Efficenza}

\textbf{L'unica cosa che si può dire è il fatto che bisogna fare molti conti (o polinomi) che crescono man mano che l'universo si allarga.}

Quando dobbiamo generare le chiavi private per ogni albero d'accesso, siamo obbligati a costruire polinomi con particolari regole.\\


\section{Breaking the code}

Per poter effettuare un attacco allo schema ABE, è necessario:

\begin{itemize}
	\item La mappa bilineare $e$. Può risultare molto complesso ricavare una formula analitica.\\ 
	È fondamentale per rompere lo schema poiché abbiamo $e(g,g) = g^\prime$ dove $g$ è un generatore del gruppo su cui si basano le chiavi mentre $g^\prime$ genera il gruppo su cui vengono criptati i messaggi.\\
	Conoscendo $g^\prime$ ci è possibile ricavare $g$ ad esempio risolvendo $e(x,x) = g^\prime$.
	\item Dalla chiave pubblica \pk, ci viene fornito $T_i = g^{t_i}$.\\
	Una volta trovato $g$ dalla mappa $e$, è possibile effettuare il logaritmo discreto per ottenere i valori $t_i$.
\end{itemize}

Una volta ottenuta la chiave principale \mk, si è in grado di decifrare qualsiasi messaggio.\\
La complessità della mappa bilineare gioca un fondamentale ruolo nella sicurezza dello schema ABE poiché maschera nella proprietà $e(g,g) = g^\prime$ entrambi i generatori utilizzati.

