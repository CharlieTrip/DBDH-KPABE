\chapter{Conclusioni}

\section{Sicurezza ed efficenza}

L'enorme quantità di conti necessari per creare tutte le chiavi private, la necessita di utilizzare $S_x$ (\ref{insiemedecript}) che è arbitrario per definizione, comporta una complessità non indifferente, tutta lineare rispetto ai parametri $\gamma$ su cui costruiamo chiavi.\\

\section{Breaking the code}

Da definizione delle chiavi publiche (\ref{pubchiavi}) si potrebbe risalire al gruppo $G_1$ con un metodo dei logaritmi discreti.\\
Per la costruzione della mappa bilineare, si può effettuare un analisi con il metodo dei logaritmi discreti anche su $G_2$ (usando i $e(g,g)$ e quel che si ottiene dalla mappa).\\
La ``sicurezza'' nasce dal fatto che i pairing non banali utilizzati sono su curve elittiche il che rende il processo di logaritmo piuttosto più complesso rispetto all'usare $\mathbb{C}_p$ e un generatore.

\textbf{Non saprei quanto si riesce a dire in più su questa parte.}