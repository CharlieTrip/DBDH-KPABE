\chapter{Considerazioni conclusive}



\section{Efficienza}
Come definito in \cite[4.3]{kpabe}, lo schema risulta complesso in maniera lineare rispetto alla grandezza del gruppo $G_1$ con una nota particolare sulla funzione di decrittazione.\\
Come descritto in \ref{insiemedecript}, dobbiamo scegliere un insieme $S_x$ di nodi su cui effettueremo la decifrazione.\\
La possibilità di scegliere questo insieme ci permettere di diminuire il numero di conti effettuati: vengono scelti i nodi con indice minore così da facilitare il calcolo dei coefficenti di Lagrange.\\
In ugual modo, conviene raccogliere i coefficenti di Lagrange per eseguire il minor numero di esponenziali possibili.


\section{Sicurezza}

Le applicazioni sono poche è principalmente orientate all'analisi della sicurezza che uno schema di questo tipo fornisce.\\
Nonostante sia piuttosto articolato violare il sistema, alcuni anni fa \cite{nict} è stato \emph{bucato} un sistema molto simile a quello presentato in questa tesi. Il motivo principale è che è ancora prematuro valutare il livello di sicurezza di un metodo crittografico basato sul pairing tra cui è presente la famiglia ABE.

\section{Analogie nella famiglia}

Durante lo studio del paper \cite{kpabe}, sono stati trovati altri paper, come \cite{sssth} o \cite{benoit}, che trattavano lo stesso problema con delle piccole variazioni:
\begin{itemize}
	\item \textbf{Identity Based Encryption} : Hanno una struttura molto simile a quella ABE con la differenza che l'attributo viene sostituito dall'identità dell'utente. Questo comporta che la propria identità, figurata come un indirizzo IP, una mail o qualcosa di simile, diventa la policy di decodifica.\\
	Questo comporta la necessità di avere un ente che garantisce e certifica l'identità della persone che utilizzano il sistema.
	\item \textbf{Hash degli attributi} : alcune applicazioni \cite{kpabe2} sulla famiglia dei Ciphertext-Policy ABE, presentano una funzione di hash degli attributi che vengono resi pubblici.\\  In questo modo è impossibile per l'utente apprendere quali sono gli attributi del sistema perché :
	\begin{itemize}
		\item Gli attributi che l'utente ha sono conosciuti sia in chiaro che hashati
		\item Gli attributi che l'utente non ha sono conosciuti solo nella forma hashata
	\end{itemize}
	Nei Key-Policy ABE, è necessario conoscere l'attributo in chiaro del messaggio cifrato per poter eseguire la verifica rispetto alle policy.
	\item \textbf{Spostare il calcolo} : alcuni paper spostano il calcolo tra i vari algoritmi. Nonostante la differenza di forma, la struttura rimane invariata.
\end{itemize}