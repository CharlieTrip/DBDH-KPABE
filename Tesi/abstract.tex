


\vspace*{\fill}
\begin{center}

\begin{minipage}{0.9\textwidth}
\textbf{Abstract}\\
Lo scopo della tesi è di analizzare uno schema di crittografia basato sul pairing tra gruppi ciclici che permetta granularità in fase di decrittazione. È stato studiato il problema considerando gli attributi come metodo per classificare i messaggi e su cui poter costruire una struttura d'accesso capace di differenziare i diversi destinatari fornendo loro una chiave privata che decifrasse unicamente i messaggi con determinati attributi. Una volta definita la struttura generale, sono stati sviluppati i vari algoritmi per poter dimostrare l'assunzione bilineare decisionale Diffie-Hellman.\\
È stato descritto un semplice esempio funzionante del metodo con la quale è stata creata un'interfaccia interattiva con cui è possibile vedere i vari passaggi del metodo. Alla fine sono stati inseriti dei commenti sull'efficacia e sulla adoperabilità del metodo nella realtà.
\end{minipage}

\end{center}
\vfill % equivalent to \vspace{\fill}